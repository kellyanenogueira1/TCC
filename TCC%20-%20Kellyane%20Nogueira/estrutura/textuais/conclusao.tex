% CONCLUSÃO--------------------------------------------------------------------

\chapter{CONCLUSÃO}
\label{chap:conclusao}

Este capítulo tem o intuito de apresentar as contribuições do trabalho para a área de pesquisa e as possíveis metas a serem alcançadas.

\section{TRABALHOS FUTUROS}
\label{sec:trabalhosFuturos}

A intenção a longo prazo é ampliar o público-alvo deste projeto de forma que o mesmo sirva como modelo para instituições semelhantes de outros Campus, podendo auxiliar o trabalho de uma maior quantidade de profissionais e na assistência adequada aos estudantes. 

\section{CONSIDERAÇÕES FINAIS}
\label{sec:consideracoesFinais}
Tendo em vista o período dificultoso em que está sendo realizado este projeto de pesquisa, em meio à uma pandemia e isolamento social, é significativo que este projeto também tenha sua parcela de contribuição para uma assistência à distância em casos de situações posteriores semelhantes a estas. 


\chapter{CRONOGRAMA}
\label{chap:cronograma}

Este capítulo apresenta o andamento e a estimativa de progresso durante o andamento do projeto, detalhando as etapas que serão efetuadas para atingir os objetivos do desta pesquisa.

\begin{quadro}[!htb]
    \centering
    \caption{Cronograma de andamento do projeto 2020.2 - 2021.1
    \label{qua:quadro-exemplo1}}
    \begin{tabular}{|p{3cm}|p{1cm}|p{1cm}|p{1cm}|p{1cm}|p{1cm}|p{1cm}|p{1cm}|p{1cm}|p{1cm}|}
        \hline
        \textbf{Etapas} & \textbf{Jun} & \textbf{Jul} & \textbf{Ago} & \textbf{Set} & \textbf{Out} & \textbf{Nov} & \textbf{Dez} & \textbf{Jan}  & \textbf{Fev}\\
        
        \hline
        Escolha do tema & X &&&&&&&&\\
        \hline
        Revisão Bibliográfica & X & X &&&&&&&\\
        \hline
        Levantamento de Requisitos &&& X &&&&&&\\
        \hline
        Modelagem do Sistema &&& X &&&&&&\\
        \hline
        Processo de prototipagem &&& X &&&&&&\\
        \hline
        Testes UX &&& X & X &&&&&\\
        \hline
        Desenvolvimento &&&& X & X & X & X & X &\\
        \hline
        Testes de Código e UI &&&&& X & X & X & X & X\\
        \hline
    \end{tabular} %\fonte{\citeonline{Barbosa2004}}
\end{quadro}