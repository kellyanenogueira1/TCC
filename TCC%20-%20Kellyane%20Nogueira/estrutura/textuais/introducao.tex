% INTRODUÇÃO-------------------------------------------------------------------

\chapter{INTRODUÇÃO}
\label{chap:introducao}
Este projeto tem em vista a grande parcela de alunos com NEEs que necessitam de atendimentos especializados no IFCE Campus Maracanaú. A responsabilidade deste atendimento é designada a profissionais como enfermeiros, pedagogos e psicólogos, que fazem o devido acompanhamento destes alunos durante todo o semestre em que estão matriculados.

A assistência não adequada para estes alunos pode acarretar no agravamento de seus problemas devido a sensação de isolamento, principalmente em casos de natureza psicológica e, em consequência, na evasão dos cursos.

\section{Objetivo Geral}
\label{sec:objGeral}
O objetivo deste projeto é produzir um sistema mobile acessível para PCDs (Pessoas com Deficiência), cuja principal tarefa é auxiliar o contato entre os alunos e os profissionais do campus dando suporte ao acompanhamento do aluno por meio do agendamento de consultas, dados de atendimento, anotações e relatórios.

A aplicação será desenvolvida de acordo com os padrões do governo federal eMAG (Modelo de Acessibilidade em Governo Eletrônico) e a HIG (Human Interfaces Guidelines) quanto a acessibilidade em dispositivos IOS.

O sistema possuirá vários perfis de acesso, de modo que o próprio aluno possa acessá-lo, os pais de alunos, profissionais que atenderão esses alunos, coordenadores de curso e direção de ensino. 

\section{Objetivos Específicos}
\label{sec:objEspecíficos}

\begin{itemize}
    \item Compreender a relação entre acessibilidade e tecnologia
    \item Cadastrar alunos PCDs e profissionais do campus
    \item Possibilitar a entrada de dados detalhados de atendimento
    \item Possibilitar o agendamento de consulta com os profissionais
    \item Disponibilizar os dados de forma organizada para análises futuras
    \item Aplicar normas de acessibilidade no sistema
\end{itemize}

