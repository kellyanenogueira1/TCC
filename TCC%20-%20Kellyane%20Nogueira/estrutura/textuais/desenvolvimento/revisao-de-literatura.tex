% REVISÃO DE LITERATURA--------------------------------------------------------

\chapter{A ACESSIBILIDADE}
\label{chap:fundamentacaoTeorica}

Determinada a proposta desta pesquisa, este capítulo tem por finalidade apresentar conceitos iniciais como o que é acessibilidade, sua importância e explorar como a tecnologia se encaixa neste contexto.

\section{A Acessibilidade e a Tecnologia}
\label{sec:}

O termo "acessibilidade" define a facilidade em se adquirir algo, entender ou usar (\citeonline{oxford2020}). Esta facilidade deve ser igualmente fornecida a todos. No entanto, com as diferenças entre as pessoas, o que é facilmente acessível para alguns pode não ser para outros como por exemplo, para as pessoas portadoras de algum tipo de deficiência:\\
\begin{citacao}
    Pessoas com deficiência são aquelas que têm impedimentos de longo prazo de natureza física, mental, intelectual ou sensorial, os quais, em interação com diversas barreiras, podem obstruir sua participação plena e efetiva na sociedade em igualdades de condições com as demais pessoas. \cite[art. 1]{decreto2009}.
\end{citacao}

A importância da acessibilidade abrange o âmbito de direito do indivíduo, têm impacto na sociedade e influencia nos negócios, pois a acessibilidade pode aprimorar a marca, impulsionar a inovação e ampliar o alcance de mercado (\citeonline{wai2017}).

Por este motivo, as formas de prover um fácil acesso de maneira igualitária têm sido discutida por muitas organizações do mundo ao longo dos anos. No Brasil, a acessibilidade é uma das obrigações gerais designada por lei:
\begin{citacao}
    Propiciar informação acessível para as pessoas com deficiência a respeito de ajudas técnicas para locomoção, dispositivos e tecnologias assistivas, incluindo novas tecnologias bem como outras formas de assistência, serviços de apoio e instalações; \cite[art. 4]{decreto2009}.
\end{citacao}

A tecnologia, atualmente, têm sua participação em várias áreas do conhecimento, desde máquinas que foram criadas para executar tarefas complexas a um aplicativo simples que que auxilia nas tarefas diárias. Dentro do contexto de acessibilidade não é diferente, o papel da tecnologia na propagação da informação e na assistência a pessoas com NEE é de grande utilidade:
\begin{citacao}
    É sabido que as novas Tecnologias da Informação e da Comunicação (TIC) vêm se tornando, de forma crescente, importantes instrumentos de nossa cultura e, sua utilização, um meio concreto de inclusão e interação no mundo (LEVY, 1999).
    Esta constatação é ainda mais evidente e verdadeira quando nos referimos a pessoas com necessidades especiais. Nestes casos, as TIC podem ser utilizadas como Tecnologia Assistiva. \cite[p. 1]{congresso2002}.
\end{citacao}

De acordo com \citeonline {WIE3079, congresso2002}, as tecnologias assistivas são ferramentas, recursos, estratégias e/ou práticas que têm por objetivo reduzir as dificuldades de PCDs promovendo mais independência, possibilitando a comunicação, auxiliando no desenvolvimento de habilidades de aprendizagem e minorizando barreiras:

\begin{citacao}
    Nesse contexto, a acessibilidade está relacionada à remoção das barreiras que impedem que mais pessoas possam perceber, compreender e usufruir de todo apoio computacional oferecido pelo ambiente computacional. \cite[p. 18]{WIE3079}.
\end{citacao}

O \cite[art. 27]{lei2015} define barreira como "qualquer entrave, obstáculo, atitude ou comportamento que limite ou impeça a participação social da pessoa, bem como o gozo, a fruição e o exercício de seus direitos à acessibilidade, à liberdade de movimento e de expressão, à comunicação, ao acesso à informação, à compreensão, à circulação com segurança, entre outros".

A \citeonline{wai2017} também argumenta sobre a remoção destas barreiras relacionadas a comunicação. De acordo com a organização, a Web pode remover barreiras encontradas por pessoas com NEE dentro do mundo físico quando se utiliza dos recursos de acessibilidade. No entanto, quando um conteúdo é produzido sem estes recursos pode agravar a situação ou até mesmo realizar o oposto, criar barreiras que impedem a interação dos usuários.

Com isto, foi criada a WCAG que disponibiliza gratuitamente muitos recursos de acessibilidade que podem tornar o conteúdo de um site ou ferramenta mais acessível. O intuito das instituições em conjunto com a W3C é padronizar formas de aplicar a acessibilidade na Web. Em janeiro de 2008, também foi publicada a primeira versão de recursos mais específicos para a acessibilidade em plataformas Mobile, onde não há variação nas diretrizes.

No Brasil, o \citeonline{emag2014} adaptou estes recursos criando um modelo de acessibilidade em governo eletrônico na justificativa de "promover a inclusão social, com distribuição de renda e diminuição das desigualdades" e argumentando que para alcançar a inclusão social é necessário que haja a inclusão digital. Em 2007, este padrão se tornou obrigatório para sítios e portais do governo brasileiro. 


\chapter{A TECNOLOGIA NO AMBIENTE EDUCACIONAL}
\label{chap:fundamentacaoTeorica}

Tendo em vista que a tecnologia pode ser usada como um grande recurso que possibilita a acessibilidade, este capítulo pretende apresentar o mesmo atuando dentro do ambiente educacional.

\section{Um problema na educação}
\label{sec:}

(\citeonline{simposio2014}), assim como muitos outros autores, demostra preocupação com o quadro educacional de PCDs, utilizando os dados do Censo Demográfico de 2010 e 2014, faz comparações de resultados significativos entre níveis de escolaridade. No quadro abaixo, uma síntese destes dados para uma melhor compreensão:
 
\begin{table}[!ht]
\caption{Comparativo entre taxas de escolaridade}
\label{tableComparacao}
\begin{tabular}{|l|c|c|}
\hline
\multicolumn{1}{|c|}{Nível de escolaridade}  & Pessoas sem deficiência & Pessoas com deficiência \\ \hline
Alfabetização (Com mais de 15 anos de idade) & 90,6\%                  & 81,7\%                  \\ \hline
Ensino Fundamental                           & 61,1\%                  & 38,2\%                  \\ \hline
Ensino Superior incompleto                   & 29,7\%                  & 17,7\%                  \\ \hline
Ensino Superior completo                     & 10,4\%                  & 6,7\%                   \\ \hline
\end{tabular}
\end{table}

É possível notar que as dificuldades encontradas por alunos com NEE podem ser um impedimento em sua formação pedagógica e profissional, requerindo uma atenção diferenciada por parte dos educadores e da gestão institucional. 

\begin{citacao}
    A educação constitui direito da pessoa com deficiência, assegurados sistema educacional inclusivo em todos os níveis e aprendizado ao longo de toda a vida, de forma a alcançar o máximo desenvolvimento possível de seus talentos e habilidades físicas, sensoriais, intelectuais e sociais, segundo suas características, interesses e necessidades de aprendizagem.  \cite[art.27]{lei2015}.
\end{citacao}

Entre os objetivos da educação escolar, encontram-se não somente os de natureza técnico-pedagógica mas também a da formação cidadã dos indivíduos. Com isto, é um grande desafio elaborar estratégias e práticas que diminuam as desigualdades de aprendizado, incluindo toda a diversidade de estudantes em diferentes níveis. No entanto, é importante ressaltar que, além de a educação básica ser um direito concedido por lei, "o êxito da integração escolar depende, dentre outros fatores, da eficiência no
atendimento e diversidade da população estudantil". (\cite[p. 24]{seesp2003})

\section{Um auxílio tecnológico}
\label{sec:}

- Explicar como a tecnologia pode ajudar no aprendizado e/ou no ambiente educacional

Além das deficiências já mencionadas neste documento, a acessibilidade tecnológica se dá também na abrangência de pessoas sem deficiência que utilizam dispositivos com diferentes tamanhos de tela e modos de entrada, idosos com debilidades comuns a sua condição, “deficiências temporárias”, como por exemplo, alguém que tenha perdido o óculos ou que por algum acidente esteja impossibilitado de usar uma das mãos e, pessoas com “situational limitations” como a baixa luminosidade e lenta conexão à internet.
A acessibilidade dá suporte para inclusão social tanto de pessoas com deficiência quanto a pessoas idosas, de áreas rurais e de países em desenvolvimento. (\citeonline{wai2017})

Entender as leis, normas e recursos da acessibilidade  é algo imprescindível quando o objetivo  é resultar em um sistema flexível e intuitivo (\citeonline{simposio2014}).


